\documentclass[
  aps,
  pre,
  reprint,
  superscriptaddress,
  nofootinbib
]{revtex4-2}

\usepackage[T1]{fontenc}
\usepackage{lmodern}
\usepackage{amsmath,amssymb,amsfonts,bm,mathtools}
\usepackage{booktabs}
\usepackage{hyperref}

\hypersetup{
  colorlinks=true,
  linkcolor=blue,
  citecolor=blue,
  urlcolor=blue
}

\newcommand{\R}{\mathbb{R}}
\newcommand{\diag}{\operatorname{diag}}

\begin{document}

\title{Diagonal Split-Mass SBP Construction by Constrained Optimization}

\author{SphericalSBPOperators Development Team}
\affiliation{Computational Science and Engineering}

\date{\today}

\begin{abstract}
We document a diagonal split-mass construction for spherical-symmetry SBP operators in which
scalar and vector moments are integrated by separate diagonal matrices. The method enforces
hard positivity of both diagonal masses and imposes symmetry, divergence, and quadrature
conditions as linear equality constraints in a constrained optimization problem. We describe
both full-constraint and divergence-first staged variants, then summarize feasibility and
constraint errors for representative runs at $p=2$ and gradient orders $d=2,4,6$.
Feasible configurations satisfy the discrete SBP identity at roundoff level.
\end{abstract}

\maketitle

\section{Problem statement}
On a mirrored uniform grid $x_i\in[-R,R]$ ($i=1,\dots,N$), let $G\in\R^{N\times N}$ be a
first-derivative SBP matrix and
\begin{equation}
B = \diag(B_{11},\dots,B_{NN}),\qquad
B_{11}=-R^p,\quad B_{NN}=R^p,
\end{equation}
with all other $B_{ii}=0$.

We seek two diagonal positive-definite mass matrices
\begin{equation}
S=\diag(s_1,\dots,s_N),\qquad V=\diag(v_1,\dots,v_N),
\end{equation}
with
\begin{equation}
s_i\ge \epsilon,\qquad v_i\ge \epsilon\quad \forall i,
\end{equation}
where $\epsilon>0$ is user-selected (default in the implementation is the next-cell volume from
$r=0$, i.e. $\epsilon=1/3$ for $\Delta r=1$).

The split SBP relation is
\begin{equation}
S D + G^T V = B,
\label{eq:split_sbp}
\end{equation}
which defines
\begin{equation}
D = S^{-1}(B-G^TV).
\label{eq:D_from_SV}
\end{equation}

\section{Constraint families}
Unknowns are collected as
\begin{equation}
z = [s_1,\dots,s_N,v_1,\dots,v_N]^T \in \R^{2N}.
\end{equation}
All imposed conditions are linear in $z$.

\subsection{Reflection symmetry}
Let $c=(N+1)/2$ be the center index. For $k=1,\dots,c-1$:
\begin{align}
s_{c-k}-s_{c+k}&=0,\\
v_{c-k}-v_{c+k}&=0.
\end{align}

\subsection{Optional anti-stiffness}
If enabled:
\begin{equation}
s_c-s_{c+1}=0,\qquad v_c-v_{c+1}=0.
\end{equation}

\subsection{Split quadrature moments}
For geometry power $p$, the target moment is
\begin{equation}
I_q = \int_{-R}^{R} r^{q}r^p\,dr.
\end{equation}

For even exponents $q\in\{0,2,\dots,q_S^{\max}\}$:
\begin{equation}
\sum_{j=1}^N s_j x_j^q = I_q.
\end{equation}

For odd exponents $q\in\{1,3,\dots,q_V^{\max}\}$:
\begin{equation}
\sum_{j=1}^N v_j x_j^q = I_q.
\end{equation}

Optionally, a $q=-1$ vector moment is also enforced:
\begin{equation}
\sum_{j=1}^N v_j \phi_j^{(-1)} = I_{-1},
\end{equation}
where $\phi_j^{(-1)}=1/x_j$ for $x_j\neq0$ and $\phi_j^{(-1)}=0$ at $x_j=0$.

\subsection{Divergence exactness on odd monomials}
For odd $k$, Eq.~\eqref{eq:split_sbp} applied to $x^k$ gives row-wise constraints:
\begin{equation}
(k+p)x_i^{k-1}s_i + \sum_{j=1}^N G_{ji}x_j^k v_j = B_{ii}x_i^k.
\label{eq:div_constraint_row}
\end{equation}

Interior rows use odd degrees up to $k\le k_{\text{int}}^{\max}$
(via \texttt{div\_max\_odd}). Boundary rows use a possibly reduced cap
$k\le k_{\text{bnd}}^{\max}$ (via \texttt{boundary\_div\_max\_odd}).

\section{Optimization model}
With active equality rows $A_{\text{act}}z=b_{\text{act}}$, we solve
\begin{align}
\min_{z\in\R^{2N}}\;& \sum_{i=1}^N (s_i-|x_i|^p)^2 + \sum_{i=1}^N (v_i-|x_i|^p)^2,\\
\text{s.t. }\;& A_{\text{act}}z=b_{\text{act}},\\
& s_i\ge\epsilon,\;v_i\ge\epsilon.
\end{align}
Ipopt is used with tight feasibility tolerances.

\paragraph{Feasibility criterion.}
A run is marked feasible if termination is successful and
\begin{equation}
\max |A_{\text{act}}z-b_{\text{act}}| \le \tau,
\end{equation}
with $\tau=\max(10^{-9},10\,\texttt{constr\_viol\_tol})$.

\section{Constraint application strategies}
Two modes are implemented.

\subsection{Full mode}
All selected constraints are active simultaneously.

\subsection{Staged quadrature mode}
First, solve with symmetry+divergence rows only.
Then attempt quadrature rows one-by-one (ordered by exponent and family), retaining only rows that
preserve feasibility under positivity bounds.

This mode prioritizes divergence constraints when full systems are difficult.

\section{Run design choices}
All reported runs use:
\begin{itemize}
  \item geometry $p=2$,
  \item mirrored grid $x=-10:1:10$ ($N=21$),
  \item odd divergence monomials in the interior according to each test,
  \item hard positivity for both masses ($S,V\succeq \epsilon I$ in diagonal form).
\end{itemize}

For $d=2,4$, we tested representative split quadrature settings and observed feasible solves with
small residuals.
For $d=6$, we focused on the case requested by the study:
interior $\{D r, D r^3\}$ and reduced boundary divergence order.

\section{Summary of feasible runs}
Table~\ref{tab:feasible_summary} reports feasible representative runs.
The quadrature columns show maximum absolute per-moment errors over imposed moments.
The divergence columns report maximum residual of imposed divergence constraints, plus
origin and boundary subsets.

\begin{table*}[t]
\centering
\caption{Feasible representative runs for diagonal split masses.}
\label{tab:feasible_summary}
\begin{tabular}{@{}llllllllll@{}}
\toprule
Case & $d$ & $\epsilon$ & $k_{\text{bnd}}^{\max}$ & Quadrature set & $\max|A z-b|$ & $\max E_S$ & $\max E_V$ & $\max E_{\text{div}}$ & $(E_{\text{div}}(0),E_{\text{div}}(\partial))$ \\
\midrule
A & 2 & $1/3$ & 1 & $S:\{0,2\},\;V:\{-1,1,3\}$ & $1.46\times 10^{-11}$ & $2.27\times 10^{-13}$ & $1.46\times 10^{-11}$ & $1.14\times 10^{-13}$ & (n/a, $1.14\times 10^{-13}$) \\
B & 4 & $1/3$ & 1 & $S:\{0,2\},\;V:\{-1,1,3\}$ & $2.91\times 10^{-11}$ & $1.14\times 10^{-13}$ & $2.18\times 10^{-11}$ & $1.22\times 10^{-13}$ & ($1.22\times 10^{-13}$, $1.14\times 10^{-13}$) \\
C & 6 & $10^{-2}$ & 1 & $S:\{0,2\},\;V:\{-1,1,3\}$ & $1.46\times 10^{-11}$ & $0$ & $7.28\times 10^{-12}$ & $4.55\times 10^{-13}$ & ($2.72\times 10^{-13}$, $2.27\times 10^{-13}$) \\
D & 6 & $10^{-3}$ & 1 & $S:\{0,2\},\;V:\{-1,1,3\}$ & $1.46\times 10^{-11}$ & $0$ & $1.09\times 10^{-11}$ & $3.04\times 10^{-13}$ & ($3.04\times 10^{-13}$, $7.11\times 10^{-14}$) \\
E & 6 & $10^{-2}$ & 0 & $S:\{0,2\},\;V:\{-1,1,3\}$ & $1.42\times 10^{-13}$ & $0$ & $0$ & $1.42\times 10^{-13}$ & ($5.55\times 10^{-17}$, n/a) \\
F & 6 & $10^{-3}$ & 0 & $S:\{0,2\},\;V:\{-1,1,3\}$ & $7.28\times 10^{-12}$ & $2.27\times 10^{-13}$ & $2.91\times 10^{-11}$ & $2.84\times 10^{-13}$ & ($1.67\times 10^{-16}$, n/a) \\
\bottomrule
\end{tabular}
\end{table*}

\paragraph{SBP residual on feasible runs.}
For feasible cases, the matrix residual
\begin{equation}
R_{\text{SBP}} = S D + G^T V - B
\end{equation}
is at roundoff level. For the $d=6$ feasible runs above, we observed
$\max|R_{\text{SBP}}|\in[7.1\times10^{-15},2.8\times10^{-14}]$.

\section{What caused infeasibility in difficult $d=6$ runs?}
For $d=6$ with interior $\{D r,D r^3\}$ and boundary $D r$ under the default
$\epsilon=1/3$, the divergence-first base system can be linearly consistent
(so not a pure rank mismatch), but still incompatible with strict positivity.
Lowering $\epsilon$ to $10^{-3}$ or $10^{-2}$ restores feasibility for the tested
quadrature set.

When boundary divergence is further tightened, genuine linear inconsistency
(rank mismatch) can also appear.

\section{Practical procedure}
A practical workflow for diagonal split masses is:
\begin{enumerate}
  \item Choose $(d,p)$ and divergence targets first (interior and boundary separately).
  \item Solve divergence+symmetry with hard positivity.
  \item If feasible, add quadrature moments (full or staged mode).
  \item Report per-family residuals and SBP matrix residual.
\end{enumerate}

\section{Reproducibility}
All runs are produced with
\texttt{scripts/diag\_mass\_qp/diag\_mass\_qp\_core.jl}
and
\texttt{scripts/diag\_mass\_qp/run\_diag\_mass\_qp.jl}.

Example command (feasible $d=6$ case):
\begin{verbatim}
julia --project=. scripts/diag_mass_qp/run_diag_mass_qp.jl \
  --p 2 --d 6 --epsilon 1e-3 \
  --div-max-odd 3 --boundary-div-max-odd 1 \
  --s-quad-order 2 --v-quad-order 3 --enforce-v-neg1
\end{verbatim}

\end{document}
